% !TEX program = xelatex
%% Requires compilation with XeLaTeX or LuaLaTeX
\documentclass[10pt,xcolor={table,dvipsnames},t]{beamer}
\usepackage{biblatex}
\usepackage{caption}
\setbeamertemplate{caption}[numbered]
\addbibresource{reference.bib}
\usepackage{hyperref}
\hypersetup{ 
pdfpagemode=FullScreen,  
colorlinks=true,linkcolor=blue}
\usepackage{enumerate}

\usepackage{listings}
\usepackage{xcolor}
\usepackage{xpatch}
\usepackage{realboxes}

\definecolor{codegreen}{rgb}{0,0.6,0}
\definecolor{codegray}{rgb}{0.5,0.5,0.5}
\definecolor{codepurple}{rgb}{0.58,0,0.82}
\definecolor{backcolour}{rgb}{0.95,0.95,0.92}

\lstdefinestyle{mystyle}{
    backgroundcolor=\color{backcolour},   
    commentstyle=\color{codegreen},
    keywordstyle=\color{magenta},
    numberstyle=\tiny\color{codegray},
    stringstyle=\color{codepurple},
    basicstyle=\ttfamily\footnotesize,
    breakatwhitespace=false,         
    breaklines=true,                 
    captionpos=b,                    
    keepspaces=true,                 
    numbers=left,                    
    numbersep=5pt,                  
    showspaces=false,                
    showstringspaces=false,
    showtabs=false,                  
    tabsize=2
}

\makeatletter
\xpretocmd\lstinline{\Colorbox{backcolour}\bgroup\appto\lst@DeInit{\egroup}}{}{}
\makeatother

\lstset{style=mystyle}

% Flow chart config
\usepackage{tikz}
\usetikzlibrary{shapes.geometric, arrows}
\tikzstyle{startstop} = [rectangle, rounded corners, minimum width=3cm, minimum height=1cm,text centered, draw=black, fill=red!30]
\tikzstyle{io} = [trapezium, trapezium left angle=70, trapezium right angle=110, minimum width=3cm, minimum height=1cm, text centered, draw=black, fill=blue!30]
\tikzstyle{process} = [rectangle, minimum width=3cm, minimum height=1cm, text centered, draw=black, fill=orange!30]
\tikzstyle{decision} = [diamond, minimum width=3cm, minimum height=1cm, text centered, draw=black, fill=green!30]
\tikzstyle{arrow} = [thick,->,>=stealth]

\usetheme{UCBerkeley}

\title[Your Short Title]{STMC coding team Training}
\subtitle{Lesson 2: Variables, data types}
\author{Tsai Yun Chen}
%\institute{}
\date{\today}

\begin{document}

\begin{frame}
  \titlepage
\end{frame}

% Uncomment these lines for an automatically generated outline.
%\begin{frame}{Outline}
%  \tableofcontents
%\end{frame}

\section{Class Goal}

\begin{frame}{Goal today}

\begin{itemize}
  \item Formally introduce the concept of variable 
  \item Introduce concept of data type: \texttt{int}, \texttt{char}, \texttt{str} and \texttt{boolean}
  \item Perform basic arithmetics and complete some simple exercise
\end{itemize}

\end{frame}

\begin{frame}{A small recap}
  So What we have learnt so far?
  \begin{itemize}
    \item Asking a computer to output something -- \texttt{print}
    \item Entering something to computer -- \texttt{input}
    \item Storing something in computer -- \texttt{xxx=}
  \end{itemize}
\end{frame}

\begin{frame}{Input, process and output}
  \begin{itemize}
    \item Almost all program consist of 3 parts: Recieving inputs, processing inputs, and returning outputs
    \item For example, in a computer game, the game continuously recieve inputs like mouse clicks, keyboard typing, process them to perform actions on the character, and finally displaying the updated screen to you
    \item Today we will like to illustrate concepts surrounding these things using an example
  \end{itemize}
\end{frame}

\begin{frame}{Variables - Your handy storage box}
  \begin{itemize}
    \item To store data, a computer uses something called a \textbf{variable}
    \item A variable is a ”storage box” with a name
    \item It stores data temporarily so that the value inside can be retrieved for further processing
  \end{itemize}
  \begin{figure}
    \centering 
    \includegraphics[width=0.6\textwidth]{img/variable.png}
    \caption*{Source: \href{https://stevenpcurtis.medium.com/what-is-a-variable-3447ac1331b9}{https://stevenpcurtis.medium.com/what-is-a-variable-3447ac1331b9}}
  \end{figure}
\end{frame}

\begin{frame}{Three parts of a variable}
  \begin{itemize}
    \item A variable always consist of 3 things:
    \begin{enumerate}
      \item \textbf{Name} for us to refer later in the program 
      \item \textbf{Value} that contain what is stored
      \item \textbf{Data type} that indicate what \textit{kinds} of value is stored
    \end{enumerate}
    \item Name and value is trivial, let's look at data types
  \end{itemize}
\end{frame}

\section{Data types}
\begin{frame}{What is data types?}
  \begin{itemize}
    \item Values in computer are always stored as 0s and 1s 
    \item The same sequence of bits can represent different things according to the way we decode it!
    \item Different data type care stored and manipulate differently by computer!
    \item So important to let the computer knows what \textbf{data types} they are dealing with
  \end{itemize}
\end{frame}

\begin{frame}{Data types: Integer}
  \begin{itemize}
    \item \textbf{Integers} ($\mathbb{Z}$) are \textit{numbers without decimal points and fraction component}
    \item  Examples of integers $\{\cdots, -4, -3, -2, -1 , 0, +1, +2, +3, +4, \cdots\}$
    \item Examples of integral data: \textit{age}, \textit{no. of apples}, \textit{no. of people}
  \end{itemize}
\end{frame}

\begin{frame}{Data types: Float}
  \begin{itemize}
    \item \textbf{Floating point numbers} are \textit{numbers with decimal points}
    \item  Examples of floating point numbers: $2.5$, $0.0$, $-3.1$, $100.173$
    \item Examples of float data: \textit{temperature}, \textit{volume}, \textit{mass}, \textit{height}
  \end{itemize}
\end{frame}

\begin{frame}{Data types: Character}
  \begin{itemize}
    \item \textbf{Character} is \textit{a single alphabet/symbol}
    \item Usually quoted by \textbf{single quotes ('')}
    \item Examples of characters numbers: \texttt{'a'},\texttt{'F'},\texttt{'@'},\texttt{'0'},\texttt{'+'}
    \item Examples of character data: \textit{letter grades},\textit{biological gender},\textit{T/F}
  \end{itemize}
\end{frame}

\begin{frame}{Data types: String}
  \begin{itemize}
    \item \textbf{String} is \textit{a sequence of one or more character}
    \item Usually quoted by \textbf{double quotes ("")}
    \item Examples of string: \texttt{"hello"},\texttt{"ymchan@gmail.com"},\texttt{"1+2=3"},\texttt{"kim\_979"}
    \item Examples of string data: \textit{name}, \textit{email}, \textit{username}, \textit{password}
  \end{itemize}
\end{frame}

\begin{frame}{Data types: Boolean}
  \begin{itemize}
    \item \textbf{Boolean} is a variable either \texttt{true} or \texttt{false}
    \item Usually denote a state, or some flag
    \item Examples of boolean data: \textit{is\_married}, \textit{is\_empty}, \textit{is\_opened}, \textit{is\_running}
  \end{itemize}
\end{frame}


\begin{frame}[fragile]{Case study I}
  \begin{columns}
    \column{0.6\textwidth}
    \begin{itemize}
      \item Variable name is \textbf{myInt}
      \item The value stored is \textbf{4}
      \item The data are numbers \textit{without decimal points}, so the data type is \textbf{integer}
      \item Python Code:\\\begin{lstlisting}[language=python]
    myInt=4\end{lstlisting}
    \end{itemize}
    \column{0.4\textwidth}
    \begin{figure}
      \includegraphics[width=0.8\textwidth]{img/variable-int.png}
    \end{figure}
  \end{columns}
\end{frame}

\begin{frame}[fragile]{Case study II}
  \begin{columns}
    \column{0.6\textwidth}
    \begin{itemize}
      \item Variable name is \textbf{myChar}
      \item The value stored is \textbf{"a"}
      \item The data is a \textit{single character}; The data type is \textbf{character}
      \item Python Code:\\\begin{lstlisting}[language=python]
    myChar='a'\end{lstlisting}
    \end{itemize}
    \column{0.4\textwidth}
    \begin{figure}
      \includegraphics[width=0.8\textwidth]{img/variable-chr.png}
    \end{figure}
  \end{columns}
\end{frame}

\begin{frame}[fragile]{Case study III}
  \begin{columns}
    \column{0.6\textwidth}
    \begin{itemize}
      \item Variable name is \textbf{myString}
      \item The value stored is \textbf{"hello"}
      \item The data is a \textit{sequence of characters}; The data type is \textbf{string}
      \item Python Code:\\\begin{lstlisting}[language=python]
    myString="hello"\end{lstlisting}
    \end{itemize}
    \column{0.4\textwidth}
    \begin{figure}
      \includegraphics[width=0.8\textwidth]{img/variable-str.png}
    \end{figure}
  \end{columns}
\end{frame}

\begin{frame}[fragile]{Checking data type}
  \begin{itemize}
    \item We can also look at the data type of things using \texttt{type()} function
\begin{lstlisting}[language=python]
  >>> type(1)
  <class 'int'>

  >>> type(1.2)
  <class 'float'>

  >>> type('12')
  <class 'str'>
  
  >>> type(True)
  <class 'bool'>
\end{lstlisting}
  \end{itemize}
\end{frame}

\begin{frame}[fragile]{Checking data type}
  \begin{itemize}
    \item Similarly for variables
\begin{lstlisting}[language=python]
  >>> myName = "John"
  >>> type(myName)
  <class 'str'>

  >>> temp = 30.2
  >>> type(temp)
  <class 'float'>
\end{lstlisting}
  \end{itemize}
\end{frame}

\begin{frame}[fragile]{Checking data type}
  \begin{itemize}
    \item We can also use the \texttt{is} keyword to check if an object belongs to a type 
\begin{lstlisting}[language=python]
>>> age = 20
>>> type(age)
<class 'int'>
>>> type(age) is int
True

>>> myName  = "rs132"
>>> type(myName) is str
True
\end{lstlisting}
  \end{itemize}
\end{frame}

%\subsection{Doing Operations}
\begin{frame}[fragile]{Operator}
  \begin{itemize}
    \item We can also manipulate the value stored in the variable using various operators. For example, if the varaible is \texttt{int} or \texttt{float}:
\begin{lstlisting}[language=python]
>>> 1.0 + 2.0     # Addition
3.0
>>> 3 - 10        # Subtraction
-7
>>> 3 * 5         # Multiplication
15
>>> 4/3           # Division
1.3333333333333333
>>> 4**5          # Exponent (4**5 = 4*4*4*4*4)
1024
>>> 17//3         # Integer division, only applicable to Int
5
>>> 17%3          # Modulo/Remainder, only applicable to Int
2
\end{lstlisting}
  \end{itemize}
\end{frame}


\begin{frame}[fragile]{Doing operations with variables}
  \begin{itemize}
    \item Similarly for string, we have the \texttt{+} and \texttt{*} operations defined 
    \item Note that some operations will results in error
\begin{lstlisting}[language=python]
>>> "hello" + "bye"       # Concatenate two string 
'hellobye'
>>> "hello"*3             # Concatenate string 3 times
'hellohellohello' 

>>> "hello" * "bye"       # Make no sense
Traceback (most recent call last):
  File "<stdin>", line 1, in <module>
TypeError: cant multiply sequence by non-int of type 'str'

>>> "hello" - "bye"       # Make no sense 2
Traceback (most recent call last):
  File "<stdin>", line 1, in <module>
TypeError: unsupported operand type(s) for -: 'str' and 'str'
\end{lstlisting}
  \end{itemize}
\end{frame}

\subsection{Saving results}
\begin{frame}[fragile]{Saving results}
  \begin{itemize}
    \item Finally, we can \textbf{assign} our computational results back to string
    \item This is done using the \texttt{=} operator
\begin{lstlisting}[language=python]
>>> i = 5
>>> i + 9       # Won't change i
14
>>> i
5
>>> i = i + 9   # i = xxx Set i to be xxx
>>> i
14
\end{lstlisting}
  \end{itemize}
\end{frame}

%\section{Combining everything}

\begin{frame}[fragile]{A simple example}
  \begin{itemize}
    \item Let's see a simple example which perform the temperature conversion,
\begin{lstlisting}[language=python]
tempF = input("Enter temperature in (F): ")
tempC = 5.0*(tempF - 32.0)/9.0
print("Temperature in (C): ",tempC)
\end{lstlisting}
  \item Let's see how it works out... Ops! An error!
\begin{lstlisting}[language=bash]
>>> python3 temp.py
Enter temperature in (F): 80
Traceback (most recent call last):
  File "temp.py", line 2, in <module>
    tempC = 5.0*(tempF-32.0)/9.0
TypeError: unsupported operand type(s) for -: 'str' and 'float'
\end{lstlisting}
  \end{itemize}
\end{frame}

\begin{frame}[fragile]{What is the error?}
  \begin{itemize}
    \item What is the error? Let's look at the \textbf{error message} more closely:
  \begin{lstlisting}[language=bash]
    File "temp.py", line 2, in <module>
    tempC = 5.0*(tempF-32.0)/9.0
TypeError: unsupported operand type(s) for -: 'str' and 'float'
\end{lstlisting}
  \item According to the message, we are using the subtraction operand (\texttt{-}) to subtract \texttt{str} and \texttt{float}!
  \item But why? Isn't our input \texttt{80} a float?
  \end{itemize}
\end{frame}

\begin{frame}[fragile]{What is the error?}
  \begin{itemize}
    \item  Is it really so? In debugging, it's always good to check our assumptions because sometimes it can be wrong!
    \item Let's check by printing the type of \texttt{tempF}
\begin{lstlisting}[language=python]
tempF = input("Enter temperature in (F): ")
print('Type of tempF: ',type(tempF))
#tempC = 5.0*(tempF - 32.0)/9.0
#print("Temperature in (C): ",tempC)
\end{lstlisting}
  \end{itemize}
\end{frame}

\begin{frame}[fragile]{What is the error?}
  \begin{itemize}
    \item Let's see what it returns now:
\begin{lstlisting}[language=bash]
>>> python3 temp.py
Enter temperature in (F): 80
Type of tempF:  <class 'str'>    <---- Ops! tempF is a string!!!
\end{lstlisting}
  \end{itemize}
\end{frame}

\begin{frame}[fragile]{What is the error?}
  \begin{itemize}
    \item Turns out, the problem is the way \texttt{input} handles input. According to the \href{https://docs.python.org/3/library/functions.html#input}{documentation}:
  \end{itemize}
  \begin{figure}
    \centering
    \includegraphics[width=0.8\textwidth]{img/python-docs.png}
  \end{figure}
\end{frame}

\begin{frame}[fragile]{What is the error?}
  \begin{itemize}
    \item So basically, \texttt{input} will convert everything we input, regardless of it's original form, into string!
    \item To battle this, we need to \textit{manually} turn \texttt{str} back to \texttt{Int}
    \item This is done by wrapping \texttt{Int()} in front of \texttt{input}, i.e. from:
\begin{lstlisting}[language=python]
input("Enter temperature in (F): ")
\end{lstlisting}
to
\begin{lstlisting}[language=python]
float(input("Enter temperature in (F): "))
\end{lstlisting}
  \end{itemize}
\end{frame}

\begin{frame}[fragile]{Final code}
  \begin{itemize}
    \item The final code is then:
\begin{lstlisting}[language=python]
tempF = float(input("Enter temperature in (F): ")) 
print('Type of tempF: ',type(tempF))
tempC = 5.0*(tempF - 32.0)/9.0
print("Temperature in (C): ",tempC)
\end{lstlisting}
  \item and finally we can run!
\begin{lstlisting}[language=bash]
>>> python3 temp.py
Enter temperature in (F): 80
Type of tempF:  <class 'float'>
Temperature in (C):  26.666666666666668
\end{lstlisting}
  \end{itemize}
\end{frame}

\begin{frame}[fragile]{Some comment on type casting}
  \begin{itemize}
    \item The trick we used to solve the problem is actually called \textbf{type casting}
    \item Type casting converts data type from one kinds to another
    \item In general, to \textit{cast} a value / variable to \texttt{<data-type>}, do:
\begin{lstlisting}[language=python]
  <data-type>(<value/variable>)
\end{lstlisting}
    \item For example:
\begin{lstlisting}[language=python]
  float("1.2")    # Converts string "1.2" to float 
  int("12")       # Converts string "12" to int 
  str(1.2)        # Converts float 1.2 to string 
\end{lstlisting}
  \end{itemize}
\end{frame}

%\section{More exercise}

%\subsection{Radius from circumference}
\begin{frame}[fragile]{Let's try an exercise}
  \begin{itemize}
    \item Write a program that takes in the circumference of a circle and output the radius of the circle
    \item Recall that the circumference of a circle is calculated as $2\pi r$
    \item Now you got the circumference, how can you get back the radius?
  \end{itemize}
\end{frame}

\begin{frame}[fragile]{Answer}
  \begin{lstlisting}[language=python]
    circum=float(input('Enter the circumference: '))
    PI=3.14159265354
    r=circum/(2*PI)
    print("The radius is:",r)\end{lstlisting}
\end{frame}

% \subsection{Swapping numbers}
\begin{frame}[fragile]{Some more exercise}
  \begin{itemize}
    \item Say now you have two variables \texttt{a} and \texttt{b}, you want to swap their value
    \item but notice that the following does not work
    \begin{lstlisting}[language=python]
      a=b
      b=a\end{lstlisting}
    \item Why?
  \end{itemize}
\end{frame}

\begin{frame}[fragile]{Swapping numbers}
  \begin{itemize}
    \item The reason behind is that, after we evaluate the first line, \texttt{a} stored the value of \texttt{b}
    \item Then when evaluating the second line, \lstinline[language=python,columns=fixed]{b=a} simply put back value of \texttt{b} to \texttt{b} itself
    \item However a common trick is that we can have another variable to temporarily storing the value of \texttt{a} to avoid forgetting it
    \item Now try to complete the code.
  \end{itemize}
\end{frame}

\begin{frame}[fragile]{Answer}
  \begin{lstlisting}[language=python]
    a=int(input('Give me the first number: '))
    b=int(input('Give me the second number: '))
    temp=a
    a=b
    b=temp
    print("Now the number is swapped, a=",a,",b=",b)\end{lstlisting}
  \begin{itemize}
    \item The above trick works for any kind of variable
    \item However we have an extra variable declared, which might be problematic when the variable size is really large
    \item Can we do it without having an extra variable?
  \end{itemize}
\end{frame}

\begin{frame}[fragile]{Yes, by the power of Math}
  \begin{itemize}
    \item Unsurprisingly, yes we can do it, below is the code
    \item \begin{lstlisting}[language=python]
      a=int(input('Give me the first number: '))
      b=int(input('Give me the second number: '))
      a=a+b
      b=a-b
      a=a-b
      print("Now the number is swapped, a=",a,",b=",b)\end{lstlisting}
  \end{itemize}
\end{frame}

\begin{frame}[fragile]{Finally, Homework}
  \begin{itemize}
    \item Homework 1 is posted on the course website, namely the HW1.ipynb
    \item it contains 3 problems, sorted in ascending order of difficulty
    \item no submission is needed, the homework only serves as way for you to practice
    \item answer will be released two weeks later, on 29/10
    \item Though no submission is required, you are still encouraged to finish the homework to practice what you have learnt, so that you don't get lost in the coming lesson.
  \end{itemize}
\end{frame}

% \subsection{Length converter}
% \begin{frame}[fragile]{Length converter}
%   \begin{exampleblock}{Problem}
%     We will write something similar to the Fahrenheit to Celsius converter. Instead of converting temperature, we will write code that convert length from miles to kilometer. Write a program that read in length in miles, then output length in kilometer. (Remark: 1 miles $\approx$ 1.60934 km)
%   \end{exampleblock}
%   \begin{exampleblock}{Expected output}
% \begin{lstlisting}
% Please enter length in miles: 2.7
% Length in kilometer: 4.345218
% \end{lstlisting}
%   \end{exampleblock}
% \end{frame}

% \begin{frame}[fragile]{Average Calculator}
%   \begin{exampleblock}{Problem}
%     Write a program that takes in 5 numbers $x_1,x_2,\cdots,x_5$ and calculate the average of the numbers $\bar{x} = (x_1 + x_2 + x_3 + x_4 + x_5)/5$
%   \end{exampleblock}
%   \begin{exampleblock}{Expected output}
% \begin{lstlisting}
% 2.5
% 2.1
% 4.5
% 1.2
% 4.5
% Average of these numbers is 2.96
% \end{lstlisting}
%       \end{exampleblock}
% \end{frame}


% \subsection{Challenging: Number of hand shakes}
% \begin{frame}{Challenging: Number of hand shakes}

%   \begin{exampleblock}{Problem}
%     $n$ business people meet for lunch and shake hands with each other. How many handshakes are there? (Obviously $n\in \mathbb{Z}^+$)\\
%     (Hint: Consider the case when $n=1,2,3,4,5$, can you see a pattern?)
%   \end{exampleblock}
% \end{frame}

% \begin{frame}{Challenging: Number of hand shakes (Answer)}

%   \begin{alertblock}{Answer}
%     Let $n$ be the number of people and $S_n$ be the number of handshakes.
%     \begin{table}[]
%       \begin{tabular}{llllllll}
%       $n$ & 1 & 2 & 3 & 4 & 5 & 6 & 7\\
%       $S_n$ & 0 & 1 & 3 & 6 & 10 & 15 & 21 
%       \end{tabular}
%       \end{table}
%       Observe that $S_n = S_{n-1} + (n-1)$ for $n>1$
%   \end{alertblock}
% \end{frame}

% \begin{frame}{Challenging: Number of handshakes (Answer)}
%   Thefore:
%   \begin{align*}
%     S_n &= S_{n-1} + (n-1) \\
%     &= (S_{n-2} + (n-2)) + (n-1)\\
%     &= (S_{n-3} + (n-3)) + (n-2) + (n-1)\\
%     &= \cdots\\
%     &= (S_{2} + 2) + 3 + \cdots + (n-2) + (n-1)\\
%     &= (S_{1} + 1) + 2 + 3 + \cdots + (n-2) + (n-1)\\
%     &= \mathbf{0 + 1 + 2 + 3 + \cdots + (n-2) + (n-1)}
%   \end{align*}
  
% \end{frame}

% \begin{frame}{Challenging: Number of handshakes (Answer)}
%   \begin{equation*}
%     S_n = 0 + 1 + 2 + 3 + \cdots + (n-2) + (n-1)
%   \end{equation*}
%   \begin{itemize}
%     \item In fact, it can be shown that $S_n = n(n-1)/2$
%     \item This is because each people shakes with $n-1$ people
%     \item Since there are $n$ peoples, the number of shakes should be $n(n-1)$
%     \item But A shakes B and B shakes A are count the same, so we double counted
%     \item Therefore $S_n = n(n-1)/2$
%   \end{itemize} 
% \end{frame}


% \begin{frame}[fragile]{Challenging: Number of handshakes (Answer)}
%   Implementation in code:
% \begin{lstlisting}[language=python]
% n = input('Enter number of people: ')
% n = int(n)
% Sn = n*(n-1)//2 # // for integer division
% print('Total number of handshakes: ', Sn)
% \end{lstlisting}
% \end{frame}

\end{document}
