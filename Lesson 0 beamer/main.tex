% !TEX program = xelatex
%% Requires compilation with XeLaTeX or LuaLaTeX
\documentclass[10pt,xcolor={table,dvipsnames},t]{beamer}
\usepackage{biblatex}
\usepackage{caption}
\setbeamertemplate{caption}[numbered]
\addbibresource{reference.bib}
\usepackage{hyperref}
\hypersetup{ 
pdfpagemode=FullScreen,  
colorlinks=true,linkcolor=blue}

\usetheme{UCBerkeley}

\title[Your Short Title]{STMC HKOI Training}
\subtitle{Lesson 0: Fundemental ideas about programming}
\author{Chan Yan Mong}
%\institute{}
\date{\today}

\begin{document}

\begin{frame}
  \titlepage
\end{frame}

% Uncomment these lines for an automatically generated outline.
%\begin{frame}{Outline}
%  \tableofcontents
%\end{frame}

\section{What is programming}

\begin{frame}{What is a computer program?}

\begin{itemize}
  \item A \textbf{computer program} is a collection of instructions that can be executed by a computer to perform a specific task \cite{enwiki:Computer_program}.
  \item Similar to recipe for cooking
  \item Teaches computer what to do and how to response to input and produce outputs
  \item Basically, doing anything on a computer involves a program of some sort
\end{itemize}
%\begin{block}{Examples}
%Some examples of commonly used commands and features are included, to help you get started.
%\end{block}

\end{frame}

\begin{frame}{Examples of computer programs}
  \begin{itemize}
    \item Web browser
    \item Computer games
    \item Mobile applications 
    \item Operation system (e.g. Windows, Linux, Unix, etc.)
    \item Productivity software (e.g. Word, PowerPoint, Excel, etc.)
  \end{itemize}
\end{frame}

\begin{frame}{What is programming?}
  \begin{itemize}
    \item \textbf{Programming} is process of designing and building an executable computer program to
    accomplish a specific computing result or to perform a specific task \cite{enwiki:Computer_programming}.
    \item By doing programming, you "teaches" the computer to do some specific tasks
  \end{itemize}
\end{frame}

\section{Why programming}
\begin{frame} {Why learning programming?}
  \begin{columns}
    \column{0.6\textwidth}
    \begin{itemize}
      \item Programming is \textit{everywhere}!
      \item Science:
      \begin{itemize}
        \item Computer simulations
        \item Automatic data collection for experiments 
        \item Analysis of huge amount of data \\(Fig.~\ref{fig:eht_blackhole})
        \item Develop more efficient algorithms
        \item Develop state-of-the-art AI 
      \end{itemize}
    \end{itemize}

    \column{0.4\textwidth}
    \begin{figure}
      \includegraphics[width=0.6\textwidth]{img/blackhole.png}
      \caption{The first image of black hole is obtained by analysing over 4.5 petabyte of data \cite{westerndigital:EHT_blackhole}. An impossible task without programming (Source: \href{https://www.nasa.gov/mission_pages/chandra/news/black-hole-image-makes-history}{NASA}).}
      \label{fig:eht_blackhole}
    \end{figure}
  \end{columns}
  
\end{frame}

\begin{frame}{Why learn programming?}
  \begin{columns}
    \column{0.6\textwidth}
    \begin{itemize}
      \item Engineering
      \begin{itemize}
        \item Run code to calculate forces and stresses when designing building or aircrafts
        \item Engineering fluid simulation
        \item Game development
        \item Mobile app development
      \end{itemize}
    \end{itemize}

    \column{0.4\textwidth}
    \begin{figure}
      \includegraphics[width=0.9\textwidth]{img/euler-beam.jpg}
      \caption{Simulation of beam oscillating freely in one end (Source: \href{https://www.featool.com/tags/beam/}{FEATool})}
      \label{fig:euler_beam}
    \end{figure}
  \end{columns}
\end{frame}

\subsection{How code works}

\begin{frame}{Cool, but what is INSIDE a computer program?}
  \begin{columns}
    \column{0.6\textwidth}
    \begin{itemize}
      \item Computer programs consist of \textbf{machine code} that are made up of 0s and 1s
      \item It's groups of 0s and 1s that represent instructions directly executable by computer
      \item Machine code is the language computer "speaks"
      \item Difficult to code in machine code
    \end{itemize}
  

    \column{0.4\textwidth}
    \begin{figure}
      \includegraphics[width=0.8\textwidth]{img/machine-code.png}
      \caption{Machine code (Source: \href{https://bit.ly/3sQendj}{https://bit.ly/3sQendj})}
      \label{fig:machine_code}
    \end{figure}
  \end{columns}

\end{frame}

\begin{frame}{Yikes! Does it mean we need to code THAT?}
  \begin{columns}
    \column{0.6\textwidth}
    \begin{itemize}
      \item Short answers: No
      \item Programmers have invented \textbf{high level language} that are closer to human language but still do the job
      \item Examples: C/C++, Java, Python, Ruby, R, PHP, etc.
      \item Code written in high level language are usually called \textbf{source code}
    \end{itemize}
  

    \column{0.4\textwidth}
    \begin{figure}
      \includegraphics[width=0.9\textwidth]{img/python-code.png}
      \caption{Source code in python (Source: me)}
      \label{fig:python_code}
    \end{figure}
  \end{columns}
  
\end{frame}

\begin{frame}{Compiler: Translate source code to machine code}
  \begin{columns}
    \column{0.6\textwidth}
    \begin{itemize}
      \item Since source code are no more than a text file, computer cannot understand them
      \item We need a device that converts source code to machine comprehensible machine code
      \item That device is called a \textbf{compiler} 
      \item The compiled result is called an \textbf{executable}, which has the file extension of \texttt{.exe} in Windows
    \end{itemize}
  

    \column{0.4\textwidth}
    \begin{figure}
      \includegraphics[width=0.9\textwidth]{img/compiling.png}
      \caption{Compiling source code to executables (Source: \href{https://bit.ly/3sQendj}{https://bit.ly/3sQendj})}
      \label{fig:compiling}
    \end{figure}
  \end{columns}
  
\end{frame}

\begin{frame}{Interpreter: Line-by-line translation}

  \begin{itemize}
    \item For some programming language (like Python), the high level instructions are compiled line-by-line during runtime
    \item The software that do that is called an \textbf{interpreter} instead of a compiler
  \end{itemize}
  
\end{frame}

\begin{frame}{Let's move on to write our first program ... }
  
\end{frame}
\begin{frame}[allowframebreaks]{Reference}
  \printbibliography
  %\bibliography{./reference.bib}
\end{frame}
\end{document}
